\chapter{User Manual}
\label{usermanual}
The application has to be managed through a command line:
\begin{enumerate}
\item connect remotely via ssh: \begin{verbatim}ssh challenge@130.192.93.82\end{verbatim}
\item insert the password---login credentials will have to be requested directly from the people managing the physical board
\item after successful authentication, use the newly opened terminal
\item get root privilegies: \begin{verbatim}sudo -i\end{verbatim}
\item clone the ``PYNQ/'' directory inside the board (there are no specific requirements about the destination)
\item install both \textbf{pynq} and \textbf{cherrypy} libraries (the latter needs Python 3.6 or above)
\item navigate to ``PYNQ/UserInterface/''
\item start the backend: \begin{verbatim}python3 web_server.py\end{verbatim}
\item the challenge is now up and running at \url{http://130.192.93.82:8080/}.
\end{enumerate}
In case the logic design needs to be modified, a new bitstream will have to be generated accordingly.
The resulting files will all have to be named ``design\_1'' (as detailed in \autoref{bitstream}) and to be pushed to the board again:
\begin{verbatim}scp path/local/file challenge@130.192.93.82:PYNQ/Board_Files/gen\end{verbatim}
where the destination was previously defined during step 5.