\chapter{Background}
\label{Chapter2}
The main focus of this chapter is explaining what a Capture-The-Flag (CTF) challenge is, how it works and in which ways it can be carried out.

Capture-the-Flag events are cybersecurity competitions where each participant has to exploit one or more vulnerabilities purposefully included inside the target platform by means of personal experience, skill and knowledge about a specific field.
The final objective is to gain access to a predefined asset by means of the so-called ``flag". The latter can be defined as a unique string that is formatted in a competition-specific manner that, if captured, certifies the success in the challenge.

There can be three main types of such an event:
\begin{itemize}
    \item \textbf{Attack/Defense}, in which two teams interact with a selected system in order to (respectively) breach into the same or protect its own integrity
    \item \textbf{Jeopardy}, where all the contestants are playing agains the organizers themselves, and never communicate with each other in any way: they just have to test the vulnerable application by themselves in order to understand it and individually report its flaws
    \item \textbf{King of the Hill}: a variation of the first category in which the competing teams rally with each other with the goal of holding control of the system for the greatest amount of time possible.
\end{itemize}

Many different topics can be covered by these contests, and the individuals always have to possess a more or less advanced understanding of the intricacies they comprise. Some typical examples are: binary, crypto, forensics, hardware, miscellaneous, networking and web.
The CTF in question is jeopardy and hardware-related, being based on a vulnerable 256-bit Block Cipher that was implemented onto a remote board, accessible via its public IP (along with a dedicated port) thanks to an online webpage.