\chapter{Conclusions}
\label{Chapter6}
As a recap, the most interesting characteristic of this project is for sure its heterogeneity in terms of levels of abstraction in which it operates: starting from the lowest layer, the logic design is implemented onto the FPGA, which in turn communicates with the Python code running inside the MCU of the board.
This way, the main methods are exposed online as APIs to be called via HTTP requests, and a web application made of HTML/CSS/JavaScritpt code performs the final processing over the transmitted data in order to make it human-readable.

This CTF challenge requires the use of a real microcontroller system, and involves the knowledge of many programming (and design) languages, in such a way that a whole chain of hardware and software modules need to collaborate in an organic way.
The result is ultimately represented by a kind of ``pipe'' that conveys information from the simplest electronic elements on a remote fixed location, all the way to the screen of the end-user: wherever they are, and whatever device, browser or internet connectivity they might be using.